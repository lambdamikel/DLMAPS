\documentclass[12pt,titlepage]{article}
\usepackage{makeidx}
\usepackage{xspace}

\title{\FaCT\ Reference Manual\\Version 1.6}
\author{Ian Horrocks\\%
        {\small horrocks@cs.man.ac.uk}\\[2ex]%
        Department of Computer Science\\%
        University of Manchester\\%
        Manchester M13 9PL\\%
        United Kingdom}
\date{August, 1998}

\setlength{\parindent}{0pt}
\setlength{\parskip}{1ex plus 0.5ex minus 0.2ex}
\setlength{\topmargin}{0pt}
\setlength{\headheight}{0pt}
\setlength{\headsep}{0pt}
\setlength{\oddsidemargin}{0pt}
\setlength{\evensidemargin}{0pt}
\setlength{\textwidth}{6.26in}
\setlength{\textheight}{9.693in}

\newlength{\temp}
\newcommand{\FaCT}{FaCT}
\newcommand{\ALC}{\ensuremath{\mathcal{ALC}}\xspace}
\newcommand{\Kris}{\textsc{Kris}\xspace}
\newcommand{\Km}{\ensuremath{\mathbf{K}_{(\mathbf{m})}}\xspace}
\newcommand{\Kfour}{\ensuremath{\mathbf{K4}}\xspace}
\newcommand{\Cname}[1]{\ensuremath{\textsf{#1}}}
\newcommand{\Rname}[1]{\ensuremath{\textsf{\textsl{#1}}}}
\newcommand{\Aname}[1]{\Rname{#1}}

\newcommand{\FM}[1]{\textbf{#1}\index{\protect#1}}
\newcommand{\Arg}[1]{\textit{#1}}
\newcommand{\Key}[1]{\texttt{#1}}
\newcommand{\TT}[1]{\texttt{#1}}
\newcommand{\Top}{\texttt{*TOP*}\index{\texttt{*TOP*}}}
\newcommand{\Bot}{\texttt{*BOTTOM*}\index{\texttt{*BOTTOM*}}}
\newcommand{\Nil}{\texttt{nil}\xspace}
\newcommand{\KI}[1]{\Key{#1}\index{\protect#1}}

\newcommand{\Hdr}[2]{
\rule[2pt]{\linewidth}{1pt}\\
{\large\FM{#1}} \hfill \emph{#2} \\
\rule[5pt]{\linewidth}{1pt}%
\addcontentsline{toc}{subsubsection}{#1}}

\newenvironment{Macro}[1]%
{\Hdr{#1}{macro}%
        \vspace{-2ex}%
        \nopagebreak
        \begin{list}{}%
        {\setlength{\leftmargin}{6.5em}%
        \setlength{\labelwidth}{6.5em}%
        \setlength{\labelsep}{0in}}
        \nopagebreak}%
{\end{list}}

\newenvironment{Function}[1]%
{\Hdr{#1}{function}%
        \vspace{-2ex}%
        \nopagebreak
        \begin{list}{}%
        {\setlength{\leftmargin}{6.5em}%
        \setlength{\labelwidth}{6.5em}%
        \setlength{\labelsep}{0in}}
        \nopagebreak}%
{\end{list}}

\newenvironment{Concept}[1]%
{\Hdr{#1}{concept}%
        \nopagebreak
        \vspace{-2ex}%
        \nopagebreak
        \begin{list}{}%
        {\setlength{\leftmargin}{6.5em}%
        \setlength{\labelwidth}{6.5em}%
        \setlength{\labelsep}{0in}}}%
{\end{list}}

\newenvironment{Arglist}%
{\begin{list}{}%
        {\setlength{\leftmargin}{6em}%
        \setlength{\labelwidth}{6em}%
        \setlength{\labelsep}{0in}}}%
{\end{list}}

\makeindex

\sloppy


\begin{document}

\maketitle

\tableofcontents
\newpage

\section{Introduction}

\FaCT\ is a prototype description logic knowledge representation
system which uses an optimised tableaux subsumption algorithm to
provide complete inference for a relatively expressive concept
description language. In particular, \FaCT\ supports transitively
closed roles, a role/attribute hierarchy and general concept inclusion
axioms (implications of the form $C \Rightarrow D$ where $C$ and $D$
are arbitrary concept descriptions).

\FaCT\ is primarily intended as a tool for conceptual schema design
and ontological engineering and thus provides only concept based
reasoning services: there is a TBox but no ABox. The interface to the
TBox is designed to be broadly compatible with that of the \Kris\
system~\cite{Baader91a}.

The correspondence between \ALC\ and the propositional modal logic
\Km\ means that \FaCT\ can also be used as a highly efficient
decision procedure for both \Km\ and \Kfour~\cite{Giunchiglia96b,
Hustadt97a}.


\subsection{Obtaining \FaCT}

\FaCT\ can be obtained from the following world wide web site:
%
\begin{itemize}
\item[] \textrm{http://www.cs.man.ac.uk/mig/people/horrocks/FaCT}
\end{itemize}

After downloading the file \texttt{\FaCT.tar.gz} use the command:
%
\begin{itemize}
\item[] \texttt{gunzip \FaCT.tar.gz ; tar xvf \FaCT.tar}
\end{itemize}
%
to create a directory \texttt{\FaCT} containing the distribution
files and directories.

To install \FaCT\ follow the instructions in the file \texttt{\FaCT/README}.

\FaCT\ has been written in Common LISP and has been tested with
Allegro Common Lisp, Harlequin LispWorks and Gnu Common Lisp.



\section{Concept Descriptions}

\FaCT\ uses the same list based concept description syntax as the \Kris\
system. If \Cname{CN} is the name of a defined concept, \Rname{R} is
the name of a defined role, \Aname{A} is the name of a defined attribute
and $C,C_1,\ldots,C_n$ are concept descriptions, then the following
are also valid concept descriptions:
%
\begin{itemize}
\item[]\begin{tabular}{l}
\Cname{*TOP*}\\
\Cname{*BOTTOM*}\\
\Cname{CN}\\
(and $C_1 \ldots C_n$)\\
(or $C_1 \ldots C_n$)\\
(not $C$)\\
(some $\Rname{R}$ $C$)\\
(all $\Rname{R}$ $C$)\\
(some $\Aname{A}$ $C$)\\
(all $\Aname{A}$ $C$)
\end{tabular}
\end{itemize}

The correspondence between this form and the standard infix notation
is shown in Table~\ref{table:factsyntax}.


\begin{table}[htb!]\begin{center}\begin{tabular}{|l|l|}
\hline
\multicolumn{1}{|c}{\FaCT\ syntax}
        & \multicolumn{1}{|c|}{Standard notation}\\
\hline
\Key{*TOP*} & $\top$\\
\Key{*BOTTOM*} & $\bot$\\
(\Key{and} $C_1$ \ldots $C_n$) & $C_1 \sqcap \ldots \sqcap C_n$ \\
(\Key{or} $C_1$ \ldots $C_n$) & $C_1 \sqcup \ldots \sqcup C_n$ \\
(\Key{not} $C$) & $\neg C$ \\
(\Key{some} $R$ $C$) & $\exists R.C$ \\
(\Key{all} $R$ $C$) & $\forall R.C$ \\
\hline
\end{tabular}
\caption{\FaCT\ concept expressions}\label{table:factsyntax}
\end{center}\end{table}




\section{Function and Macro Interface}  \label{sec:interface}

This section describes the built-in concepts, macros and functions
which provide the user interface to the TBox.


\subsection{Built-in Concepts}

\begin{Concept}{\Top}

\item[\textbf{Description:}\hfill] The name of the top concept
($\top$).

\item[\textbf{Remarks:}\hfill] Every concept in the TBox is subsumed
by \Top.

\end{Concept}


\begin{Concept}{\Bot}

\item[\textbf{Description:}\hfill] The name of the bottom concept
($\bot$).

\item[\textbf{Remarks:}\hfill] Every concept in the TBox subsumes
\Bot. Note that incoherent concepts become synonyms for \Bot.

\end{Concept}



\subsection{Knowledge Base Management}

\begin{Function}{init-tkb}

\item[\textbf{Description:}\hfill] Initialises the TBox.

\item[\textbf{Syntax:}\hfill] (\FM{init-tkb})

\item[\textbf{Remarks:}\hfill] All user defined concepts, roles and
implications are deleted from the TBox leaving only \Top\ and \Bot.

\item[\textbf{Examples:}\hfill] \TT{(init-tkb)}

\end{Function}


\begin{Function}{load-tkb}

\item[\textbf{Description:}\hfill] Loads a TBox from a file.

\item[\textbf{Syntax:}\hfill] (\FM{load-tkb} \Arg{name} \Key{\&key}
(\Arg{verbose} \TT{T}) (\Arg{overwrite} \Nil))

\item[\textbf{Arguments:}\hfill]%
%
        \begin{Arglist}

        \item[\Arg{name}\hfill -~] Name of the TBox file.

        \item[\Arg{verbose}\hfill -~] Keyword which, if non-\Nil,
        causes the classifier to print symbols indicating the progress
        of the load operation: \TT{P} for each primitive concept,
        \TT{C} for each non-primitive concept, \TT{R} for each role,
        \TT{R+} for each transitive role, \TT{A} for each attribute
        and \TT{I} for each implication.  If omitted, \Arg{verbose}
        defaults to \TT{T}.

        \item[\Arg{overwrite}\hfill -~] Keyword which, if non-\Nil,
        causes the classifier to clears the existing TBox from memory
        (by performing an \FM{init-tkb}) before loading the new TBox.
        If omitted, \Arg{overwrite} defaults to \Nil.

        \end{Arglist}

\item[\textbf{Return:}\hfill] \TT{T} if the TBox is successfully
loaded; \Nil\ otherwise.

\item[\textbf{Examples:}\hfill] \TT{(load-tkb 'demo.kb :verbose T)}

\end{Function}



\subsection{TBox Definitions}

\begin{Macro}{defprimconcept}

\item[\textbf{Description:}\hfill] Defines a primitive concept.

\item[\textbf{Syntax:}\hfill] (\FM{defprimconcept} \Arg{name} \Key{\&optional}
(\Arg{description} \Top))

\item[\textbf{Arguments:}\hfill]%
        \begin{Arglist}
        \item[\Arg{name}\hfill -~] Name of the new concept.
        \item[\Arg{description}\hfill -~] Optional description of the
        new concept. If omitted it defaults to \Top.
        \end{Arglist}

\item[\textbf{Return:}\hfill] A concept structure \TT{c[}\Arg{name}\TT{]} is returned.

\item[\textbf{Remarks:}\hfill] The new concept is not classified until
a call is made to \FM{classify-tkb}. It is an error if a concept
\Arg{name} has already been defined.

\item[\textbf{Examples:}\hfill] \TT{(defprimconcept ANIMAL)} \\
        \TT{(defprimconcept MALE)} \\
        \TT{(defprimconcept FEMALE)} \\
        \TT{(defprimconcept BIPED)} \\
        \TT{(defprimconcept HUMAN (and ANIMAL BIPED))}

\end{Macro}


\begin{Function}{defprimconcept-f}

\item[\textbf{Description:}\hfill] Functional equivalent of
\FM{defprimconcept}.

\item[\textbf{Remarks:}\hfill] Note that all arguments have to be
quoted.

\item[\textbf{Examples:}\hfill] \TT{(defprimconcept-f 'ANIMAL)} \\
        \TT{(defprimconcept-f 'MALE)} \\
        \TT{(defprimconcept-f 'FEMALE)} \\
        \TT{(defprimconcept-f 'BIPED)} \\
        \TT{(defprimconcept-f 'HUMAN '(and ANIMAL BIPED))}

\end{Function}


\begin{Macro}{defconcept}

\item[\textbf{Description:}\hfill] Defines a non-primitive concept.

\item[\textbf{Syntax:}\hfill] (\FM{defconcept} \Arg{name} \Arg{description})

\item[\textbf{Arguments:}\hfill]%
        \begin{Arglist}
        \item[\Arg{name}\hfill -~] Name of the new concept.
        \item[\Arg{description}\hfill -~] Description of the new concept.
        \end{Arglist}

\item[\textbf{Return:}\hfill] A concept structure \TT{c[}\Arg{name}\TT{]} is returned.

\item[\textbf{Remarks:}\hfill] The new concept is not classified until
a call is made to \FM{classify-tkb}. It is an error if a concept
\Arg{name} has already been defined. Note that in contrast to
\FM{defprimconcept} it is also an error if
\Arg{description} is omitted.

\item[\textbf{Examples:}\hfill] \TT{(defconcept MAN (and MALE HUMAN))} \\
        \TT{(defconcept WOMAN (and FEMALE HUMAN))}

\end{Macro}


\begin{Function}{defconcept-f}

\item[\textbf{Description:}\hfill] Functional equivalent of
\FM{defconcept}.

\item[\textbf{Remarks:}\hfill] Note that all arguments have to be
quoted.

\item[\textbf{Examples:}\hfill] \TT{(defconcept-f 'MAN '(and MALE HUMAN))} \\
        \TT{(defconcept-f 'WOMAN '(and FEMALE HUMAN))}

\end{Function}


\begin{Macro}{defprimrole}

\item[\textbf{Description:}\hfill] Defines a primitive role.

\item[\textbf{Syntax:}\hfill] (\FM{defprimrole} \Arg{name}
\Key{\&key} (\Arg{supers} \Nil) (\Arg{transitive} \Nil))

\item[\textbf{Arguments:}\hfill]%

        \begin{Arglist}

        \item[\Arg{name}\hfill -~] Name of the new primitive role.

        \item[\Arg{supers}\hfill -~] Keyword list of super-roles. If
        omitted it defaults to \Nil.

        \item[\Arg{transitive}\hfill -~] Keyword which, if non-\Nil,
        makes the role transitive. If omitted it defaults to \Nil.

        \end{Arglist}

\item[\textbf{Return:}\hfill] A role structure \TT{r[}\Arg{name}\TT{]} is returned.

\item[\textbf{Remarks:}\hfill] It is an error if a role or attribute
\Arg{name} has already been defined.

\item[\textbf{Examples:}\hfill] \TT{(defprimrole Relation :transitive T)} \\
        \TT{(defprimrole Close-relation :supers (Relation))}\\
        \TT{(defprimrole Ancestor :supers (Relation) :transitive T)}\\
        \TT{(defprimrole Parent :supers (Close-relation Ancestor))}

\end{Macro}


\begin{Function}{defprimrole-f}

\item[\textbf{Description:}\hfill] Functional equivalent of
\FM{defprimrole}.

\item[\textbf{Remarks:}\hfill] Note that all arguments have to be
quoted.

\item[\textbf{Examples:}\hfill] \TT{(defprimrole-f 'Relation :transitive T)} \\
        \TT{(defprimrole-f 'Close-relation :supers '(Relation))}\\
        \TT{(defprimrole-f 'Ancestor :supers '(Relation) :transitive T)}\\
        \TT{(defprimrole-f 'Parent :supers '(Close-relation Ancestor))}

\end{Function}


\begin{Macro}{defprimattribute}

\item[\textbf{Description:}\hfill] Defines a primitive attribute
(functional role).

\item[\textbf{Syntax:}\hfill] (\FM{defprimattribute} \Arg{name}
\Key{\&key} (\Arg{supers} \Nil))

\item[\textbf{Arguments:}\hfill]%

        \begin{Arglist}

        \item[\Arg{name}\hfill -~] Name of the new primitive attribute.

        \item[\Arg{supers}\hfill -~] Keyword list of 
        super-attributes or super-roles. If omitted it defaults to
        \Nil.

        \end{Arglist}

\item[\textbf{Return:}\hfill] A role structure \TT{r[}\Arg{name}\TT{]} is returned.

\item[\textbf{Remarks:}\hfill] It is an error if a role or attribute
\Arg{name} has already been defined. Note that unlike roles,
attributes cannot be transitive.

\item[\textbf{Examples:}\hfill] \TT{(defprimattribute Best-friend)} \\
        \TT{(defprimattribute Father :supers (Parent))}\\
        \TT{(defprimattribute Mother :supers (Parent))}

\end{Macro}


\begin{Function}{defprimattribute-f}

\item[\textbf{Description:}\hfill] Functional equivalent of
\FM{defprimattribute}.

\item[\textbf{Remarks:}\hfill] Note that all arguments have to be
quoted.

\item[\textbf{Examples:}\hfill] \TT{(defprimattribute-f 'Best-friend)} \\
        \TT{(defprimattribute-f 'Father :supers '(Parent))}\\
        \TT{(defprimattribute-f 'Mother :supers '(Parent))}

\end{Function}


\begin{Macro}{implies}

\item[\textbf{Description:}\hfill] An implication/subsumption axiom
between two concept descriptions.

\item[\textbf{Syntax:}\hfill] (\FM{implies} \Arg{antecedent} \Arg{consequent})

\item[\textbf{Arguments:}\hfill]%
        \begin{Arglist}
        \item[\Arg{antecedent}\hfill -~] Description of the antecedent concept.
        \item[\Arg{consequent}\hfill -~] Description of the consequent concept.
        \end{Arglist}

\item[\textbf{Return:}\hfill] \Arg{antecedent} is returned.

\item[\textbf{Remarks:}\hfill] Asserts that \Arg{antecedent} implies
\Arg{consequent} ($\Arg{antecedent} \Rightarrow \Arg{consequent}$) or,
equivalently that the \Arg{antecedent} is subsumed by \Arg{consequent}
($\Arg{antecedent} \sqsubseteq \Arg{consequent}$). Note that adding
implications \emph{after} the TBox has been classified may cause
strange and unpredictable results.

\item[\textbf{Examples:}\hfill] \TT{(implies (and POLYGON ((some
Angles 3))) (some Sides 3))}

\end{Macro}


\begin{Function}{implies-f}

\item[\textbf{Description:}\hfill] Functional equivalent of
\FM{implies}.

\item[\textbf{Remarks:}\hfill] Note that all arguments have to be
quoted.

\item[\textbf{Examples:}\hfill] \TT{(implies-f '(and POLYGON ((some
Angles 3))) '(some Sides 3))}

\end{Function}


\begin{Macro}{disjoint}

\item[\textbf{Description:}\hfill] A disjointness axiom between
concept descriptions.

\item[\textbf{Syntax:}\hfill] (\FM{disjoint} \Arg{description-1} \Arg{description-n})

\item[\textbf{Arguments:}\hfill]%
        \begin{Arglist}
        \item[\Arg{description-i}\hfill -~] A concept description.
        \end{Arglist}

\item[\textbf{Remarks:}\hfill] Asserts that the extensions of
\Arg{description-1},\ldots,\Arg{description-n} are
disjoint.

\item[\textbf{Examples:}\hfill] \TT{(disjoint MALE FEMALE)}\\
        \TT{(disjoint CAT DOG RABBIT HAMSTER)}

\end{Macro}


\begin{Function}{disjoint-f}

\item[\textbf{Description:}\hfill] Functional equivalent of
\FM{disjoint}.

\item[\textbf{Remarks:}\hfill] Note that all arguments have to be
quoted.

\item[\textbf{Examples:}\hfill] \TT{(disjoint-f 'MALE 'FEMALE)}\\
        \TT{(disjoint 'CAT 'DOG 'RABBIT 'HAMSTER)}

\end{Function}



\subsection{TBox Inferences}


\begin{Function}{classify-tkb}

\item[\textbf{Description:}\hfill] Classifies the TBox.

\item[\textbf{Syntax:}\hfill] (\FM{classify-tkb} \Key{\&key}
(\Arg{mode} \TT{:nothing}))

\item[\textbf{Arguments:}\hfill]%
%
        \begin{Arglist}

        \item[\Arg{mode}\hfill -~] Keyword which controls the output
        from the classifier:

                \begin{Arglist}

                \item[\TT{:nothing}\hfill -~] None;

                \item[\TT{:stars}\hfill -~] A symbol for each concept
                classified: \TT{P} for a primitive concept, \TT{C}
                for a non-primitive concept and \TT{S} for a synonym;

                \item[\TT{:names}\hfill -~] The name of each concept
                classified followed by -\TT{P} for a primitive
                concept, -\TT{C} for an non-primitive concept and
                -\TT{S} for a synonym;

                \item[\TT{:count}\hfill -~] The number of subsumption
                and satisfiability tests.

                \end{Arglist}

        Warnings are always output regardless of the setting of
        \Arg{mode}. If \Arg{mode} is either \TT{:stars} or \TT{:names}
        then symbols are also output during the pre-processing of
        roles (\TT{r} for each role \TT{a} for each attribute),
        implications (\TT{p} for each implication absorbed into a
        primitive concept, \TT{i} for each non-absorbed implication)
        and concept terms (\TT{c} for each concept term). If omitted,
        \Arg{mode} defaults to \TT{:nothing}.

        \end{Arglist}
        
\item[\textbf{Return:}\hfill] Returns \TT{T} if the current TBox
is coherent (i.e., the concept \Top\ is satisfiable),
\Nil\ otherwise.

\item[\textbf{Remarks:}\hfill] If any new implications have been added
to the TBox (\FM{implies} or \FM{f-implies}) since the last TBox
classification (\FM{classify-tkb}), all concepts will be reclassified;
otherwise classification will be incremental.

\item[\textbf{Examples:}\hfill] \TT{(classify-tkb :mode :stars)}

\end{Function}


\begin{Function}{direct-supers}

\item[\textbf{Description:}\hfill] Finds the direct super-concepts of
a classified concept.

\item[\textbf{Syntax:}\hfill] (\FM{direct-supers} \Arg{name})

\item[\textbf{Arguments:}\hfill]%
        \begin{Arglist}
        \item[\Arg{name}\hfill -~] A concept name.
        \end{Arglist}

\item[\textbf{Return:}\hfill] Returns a list of the direct subsumers
(super-concepts) of the concept \Arg{name}.

\item[\textbf{Remarks:}\hfill] The TBox must have been classified
using \FM{classify-tkb}. Note that \Arg{name} must be quoted.

\item[\textbf{Examples:}\hfill] \TT{(direct-supers 'MAN)}
$\Rightarrow$ \TT{(c[HUMAN] c[MALE])}

\end{Function}


\begin{Function}{all-supers}
%
\item[\textbf{Description:}\hfill] Finds all the super-concepts of
a classified concept.

\item[\textbf{Syntax:}\hfill] (\FM{all-supers} \Arg{name})

\item[\textbf{Arguments:}\hfill]%
        \begin{Arglist}
        \item[\Arg{name}\hfill -~] A concept name.
        \end{Arglist}

\item[\textbf{Return:}\hfill] Returns a list of all the subsumers
(super-concepts) of the concept \Arg{name}.

\item[\textbf{Remarks:}\hfill] The TBox must have been classified
using \FM{classify-tkb}. Note that \Arg{name} must be quoted.

\item[\textbf{Examples:}\hfill] \TT{(all-supers 'MAN)}
$\Rightarrow$ \TT{(c[ANIMAL] c[HUMAN] c[BIPED] c[MALE] c[*TOP*])}

\end{Function}


\begin{Function}{direct-subs}

\item[\textbf{Description:}\hfill] Finds the direct sub-concepts of
a classified concept.

\item[\textbf{Syntax:}\hfill] (\FM{direct-subs} \Arg{name})

\item[\textbf{Arguments:}\hfill]%
        \begin{Arglist}
        \item[\Arg{name}\hfill -~] A concept name.
        \end{Arglist}

\item[\textbf{Return:}\hfill] Returns a list of the direct subsumees
(sub-concepts) of the concept \Arg{name}.

\item[\textbf{Remarks:}\hfill] The TBox must have been classified
using \FM{classify-tkb}. Note that \Arg{name} must be quoted.

\item[\textbf{Examples:}\hfill] \TT{(direct-subs 'MALE)}
$\Rightarrow$ \TT{(c[MAN])}

\end{Function}


\begin{Function}{all-subs}

\item[\textbf{Description:}\hfill] Finds all the direct sub-concepts of
a classified concept.

\item[\textbf{Syntax:}\hfill] (\FM{all-subs} \Arg{name})

\item[\textbf{Arguments:}\hfill]%
        \begin{Arglist}
        \item[\Arg{name}\hfill -~] A concept name.
        \end{Arglist}

\item[\textbf{Return:}\hfill] Returns a list of all the subsumees
(sub-concepts) of the concept \Arg{name}.

\item[\textbf{Remarks:}\hfill] The TBox must have been classified
using \FM{classify-tkb}. Note that \Arg{name} must be quoted.

\item[\textbf{Examples:}\hfill] \TT{(all-subs 'MALE)}
$\Rightarrow$ \TT{(c[MAN] c[*BOTTOM*])}

\end{Function}


\begin{Function}{equivalences}

\item[\textbf{Description:}\hfill] Finds those concepts which are
equivalent to a classified concept.

\item[\textbf{Syntax:}\hfill] (\FM{equivalences} \Arg{name})

\item[\textbf{Arguments:}\hfill]%
        \begin{Arglist}
        \item[\Arg{name}\hfill -~] A concept name.
        \end{Arglist}

\item[\textbf{Return:}\hfill] Returns a list of all the concepts which
are equivalent to (synonyms for) \Arg{name}.

\item[\textbf{Remarks:}\hfill] The TBox must have been classified
using \FM{classify-tkb}. Note that \Arg{name} must be quoted.

\end{Function}


\begin{Function}{satisfiable}

\item[\textbf{Description:}\hfill] Tests if a concept description is satisfiable.

\item[\textbf{Syntax:}\hfill] (\FM{satisfiable} \Arg{description})

\item[\textbf{Arguments:}\hfill]%
        \begin{Arglist}
        \item[\Arg{description}\hfill -~] A concept description.
        \end{Arglist}

\item[\textbf{Return:}\hfill] Returns \TT{T} if \Arg{description} is
satisfiable w.r.t.\ the current TBox, \Nil\ otherwise.

\item[\textbf{Remarks:}\hfill] The TBox must have been classified
using \FM{classify-tkb}. Note that \Arg{description} must be quoted.

\item[\textbf{Examples:}\hfill] \TT{(satisfiable '(and MALE FEMALE))}
$\Rightarrow$ \Nil\\
\TT{(satisfiable '(and MALE ANIMAL))}
$\Rightarrow$ \TT{T}

\end{Function}


\begin{Function}{subsumes}

\item[\textbf{Description:}\hfill] Tests if one concept description
subsumes another.

\item[\textbf{Syntax:}\hfill] (\FM{subsumes} \Arg{description-1} \Arg{description-2})

\item[\textbf{Arguments:}\hfill]%
        \begin{Arglist}
        \item[\Arg{description-1}\hfill -~] A concept description.
        \item[\Arg{description-2}\hfill -~] A concept description.
        \end{Arglist}

\item[\textbf{Return:}\hfill] Returns \TT{T} if \Arg{description-1}
subsumes \Arg{description-2} w.r.t.\ the current TBox, \Nil\ otherwise.

\item[\textbf{Remarks:}\hfill] The TBox must have been classified
using \FM{classify-tkb}. Note that \Arg{description-1} and
\Arg{description-2} must be quoted.

\item[\textbf{Examples:}\hfill] \TT{(subsumes '(and MALE ANIMAL) 'MAN)}
$\Rightarrow$ \TT{T}

\end{Function}


\begin{Function}{equivalent-concepts}

\item[\textbf{Description:}\hfill] Tests if two concept descriptions
are equivalent.

\item[\textbf{Syntax:}\hfill] (\FM{equivalent-concepts}
\Arg{description-1} \Arg{description-2})

\item[\textbf{Arguments:}\hfill]%
        \begin{Arglist}
        \item[\Arg{description-1}\hfill -~] A concept description.
        \item[\Arg{description-2}\hfill -~] A concept description.
        \end{Arglist}

\item[\textbf{Return:}\hfill] Returns \TT{T} if \Arg{description-1} is
equivalent to \Arg{description-2} w.r.t.\ the current TBox, \Nil\ 
otherwise.

\item[\textbf{Remarks:}\hfill] The TBox must have been classified
using \FM{classify-tkb}. Note that \Arg{description-1} and
\Arg{description-2} must be quoted.

\item[\textbf{Examples:}\hfill] \TT{(equivalent-concepts '(and MALE HUMAN) 'MAN)}
$\Rightarrow$ \TT{T}\\
\TT{(equivalent-concepts 'HUMAN '(and ANIMAL BIPED))}
$\Rightarrow$ \Nil

\end{Function}


\begin{Function}{disjoint-concepts}

\item[\textbf{Description:}\hfill] Tests if two concept descriptions
are disjoint.

\item[\textbf{Syntax:}\hfill] (\FM{disjoint-concepts}
\Arg{description-1} \Arg{description-2})

\item[\textbf{Arguments:}\hfill]%
        \begin{Arglist}
        \item[\Arg{description-1}\hfill -~] A concept description.
        \item[\Arg{description-2}\hfill -~] A concept description.
        \end{Arglist}

\item[\textbf{Return:}\hfill] Returns \TT{T} if \Arg{description-1} is
disjoint from \Arg{description-2} w.r.t.\ the current TBox, \Nil\ 
otherwise.

\item[\textbf{Remarks:}\hfill] The TBox must have been classified
using \FM{classify-tkb}. Note that \Arg{description-1} and
\Arg{description-2} must be quoted.

\item[\textbf{Examples:}\hfill] \TT{(disjoint-concepts 'MALE 'FEMALE)}
$\Rightarrow$ \TT{T}

\end{Function}


\begin{Function}{classify-concept}

\item[\textbf{Description:}\hfill] Finds where a concept description
would classify without adding it to the TBox.

\item[\textbf{Syntax:}\hfill] (\FM{classify-concept} \Arg{description})

\item[\textbf{Arguments:}\hfill]%
        \begin{Arglist}
        \item[\Arg{description}\hfill -~] A concept description.
        \end{Arglist}

\item[\textbf{Return:}\hfill] Returns three values: a list of all the
direct-supers of \Arg{description}; a list of all the direct-subs of
\Arg{description}; a list of all the concepts which are equivalent
to \Arg{description}.

\item[\textbf{Remarks:}\hfill] The TBox must have been classified
using \FM{classify-tkb}. Note that \Arg{description} must be quoted.

\item[\textbf{Examples:}\hfill] \TT{(classify-concept '(and MALE ANIMAL))}
$\Rightarrow$ \TT{(c[MALE] c[ANIMAL]) (c[MAN])} \Nil

\end{Function}


\begin{Macro}{add-concept}

\item[\textbf{Description:}\hfill] Defines a new concept and
classifies the TBox using \FM{classify-tkb}.

\item[\textbf{Syntax:}\hfill] (\FM{add-concept} \Arg{name}
\Arg{description} \Key{\&key} (\Arg{primitive} \Nil))

\item[\textbf{Arguments:}\hfill]%
        \begin{Arglist}
        \item[\Arg{name}\hfill -~] Name of the new concept.
        \item[\Arg{description}\hfill -~] Description of the new concept.
        \item[\Arg{primitive}\hfill -~] Keyword which, if non-\Nil,
        makes the concept primitive. If omitted it defaults to \Nil.
        \end{Arglist}

\item[\textbf{Return:}\hfill] A concept structure \TT{c[}\Arg{name}\TT{]} is returned.

\item[\textbf{Remarks:}\hfill] It is an error if a concept \Arg{name}
has already been defined.

\item[\textbf{Examples:}\hfill] \TT{(add-concept VEGETABLE \Top\
        :primitive T)} \\
        \TT{(add-concept WOMAN (and FEMALE HUMAN))}

\end{Macro}


\begin{Function}{add-concept-f}

\item[\textbf{Description:}\hfill] Functional equivalent of
\FM{add-concept}.

\item[\textbf{Remarks:}\hfill] Note that all arguments have to be
quoted.

\item[\textbf{Examples:}\hfill] \TT{(add-concept-f 'VEGETABLE '\Top\ 
        :primitive T)} \\
        \TT{(add-concept-f 'WOMAN '(and FEMALE HUMAN))}

\end{Function}



\subsection{TBox Queries}

\begin{Function}{get-concept}

\item[\textbf{Description:}\hfill] Retrieves a concept from the TBox.

\item[\textbf{Syntax:}\hfill] (\FM{get-concept} \Arg{name})

\item[\textbf{Arguments:}\hfill]%
        \begin{Arglist}
        \item[\Arg{name}\hfill -~] Concept name (a LISP atom).
        \end{Arglist}

\item[\textbf{Return:}\hfill] A concept structure
\TT{c[}\Arg{name}\TT{]} if a concept \Arg{name} is defined in the
TBox; \Nil\ otherwise.

\end{Function}


\begin{Function}{get-role}

\item[\textbf{Description:}\hfill] Retrieves a role or attribute
from the TBox.

\item[\textbf{Syntax:}\hfill] (\FM{get-role} \Arg{name})

\item[\textbf{Arguments:}\hfill]%
        \begin{Arglist}
        \item[\Arg{name}\hfill -~] Role or attribute name (a LISP atom).
        \end{Arglist}

\item[\textbf{Return:}\hfill] A role structure \TT{r[}\Arg{name}\TT{]}
if a role or attribute \Arg{name} is defined in the TBox; \Nil\
otherwise.

\end{Function}


\begin{Function}{get-all-concepts}

\item[\textbf{Description:}\hfill] Retrieves all concepts from the TBox.

\item[\textbf{Syntax:}\hfill] (\FM{get-all-concepts})

\item[\textbf{Return:}\hfill] A list of all the concepts defined in
the TBox.

\end{Function}


\begin{Function}{get-all-roles}

\item[\textbf{Description:}\hfill] Retrieves all roles and attributes from the TBox.

\item[\textbf{Syntax:}\hfill] (\FM{get-all-roles})

\item[\textbf{Return:}\hfill] A list of all the roles and attributes
defined in the TBox.

\end{Function}


\begin{Function}{classified-tkb?}

\item[\textbf{Description:}\hfill] Tests if the TBox is classified.

\item[\textbf{Syntax:}\hfill] (\FM{classified-tkb?})

\item[\textbf{Return:}\hfill] \TT{T} if the TBox is classified; \Nil\ otherwise.

\end{Function}


\begin{Function}{what-is?}

\item[\textbf{Description:}\hfill] Determines the type of a concept,
or role structure.

\item[\textbf{Syntax:}\hfill] (\FM{what-is?} \Arg{structure})

\item[\textbf{Arguments:}\hfill]%
        \begin{Arglist}
        \item[\Arg{structure}\hfill -~] Concept or role structure.
        \end{Arglist}

\item[\textbf{Return:}\hfill] The type of structure, one of
\TT{CONCEPT}, \TT{PRIMITIVE}, \TT{ROLE} or \TT{FEATURE}.

\end{Function}


\begin{Function}{is-primitive?}

\item[\textbf{Description:}\hfill] Determines if a structure
is a primitive concept (of type \TT{PRIMITIVE}).

\item[\textbf{Syntax:}\hfill] (\FM{is-primitive?} \Arg{structure})

\item[\textbf{Arguments:}\hfill]%
        \begin{Arglist}
        \item[\Arg{structure}\hfill -~] Concept structure.
        \end{Arglist}

\item[\textbf{Return:}\hfill] \TT{T} if \Arg{structure} is a primitive
concept (of type \TT{PRIMITIVE}); \Nil\ otherwise.

\end{Function}


\begin{Function}{is-concept?}

\item[\textbf{Description:}\hfill] Determines if a structure
is a non-primitive concept (of type \TT{CONCEPT}).

\item[\textbf{Syntax:}\hfill] (\FM{is-concept?} \Arg{structure})

\item[\textbf{Arguments:}\hfill]%
        \begin{Arglist}
        \item[\Arg{structure}\hfill -~] Concept structure.
        \end{Arglist}

\item[\textbf{Return:}\hfill] \TT{T} if \Arg{structure} is a non-primitive
concept (of type \TT{CONCEPT}); \Nil\ otherwise.

\end{Function}


\begin{Function}{is-role?}

\item[\textbf{Description:}\hfill] Determines if a structure
is a role (of type \TT{ROLE}).

\item[\textbf{Syntax:}\hfill] (\FM{is-role?} \Arg{structure})

\item[\textbf{Arguments:}\hfill]%
        \begin{Arglist}
        \item[\Arg{structure}\hfill -~] Role structure.
        \end{Arglist}

\item[\textbf{Return:}\hfill] \TT{T} if \Arg{structure} is a role
(of type \TT{ROLE}); \Nil\ otherwise.

\end{Function}


\begin{Function}{is-feature?}

\item[\textbf{Description:}\hfill] Determines if a structure
is an attribute (of type \TT{FEATURE}).

\item[\textbf{Syntax:}\hfill] (\FM{is-feature?} \Arg{structure})

\item[\textbf{Arguments:}\hfill]%
        \begin{Arglist}
        \item[\Arg{structure}\hfill -~] Role structure.
        \end{Arglist}

\item[\textbf{Return:}\hfill] \TT{T} if \Arg{structure} is an attribute
(of type \TT{FEATURE}); \Nil\ otherwise.

\end{Function}


\begin{Function}{name}

\item[\textbf{Description:}\hfill] Retrieves the name of a concept or
role structure.

\item[\textbf{Syntax:}\hfill] (\FM{name} \Arg{structure})

\item[\textbf{Arguments:}\hfill]%
        \begin{Arglist}
        \item[\Arg{structure}\hfill -~] Concept or role structure.
        \end{Arglist}

\item[\textbf{Return:}\hfill] The name of \Arg{structure} if it is a
concept or role structure; \Nil\ otherwise.

\end{Function}


\begin{Function}{description}

\item[\textbf{Description:}\hfill] Retrieves the description of a
concept or role structure.

\item[\textbf{Syntax:}\hfill] (\FM{description} \Arg{structure})

\item[\textbf{Arguments:}\hfill]%
        \begin{Arglist}
        \item[\Arg{structure}\hfill -~] Concept or role structure.
        \end{Arglist}

\item[\textbf{Return:}\hfill] The description of \Arg{structure} if it is a
concept or role structure; \Nil\ otherwise. The format of the
description depends on the type of structure:
%
        \begin{Arglist}
%
        \item[\TT{PRIMITIVE}\hfill -~] The \Arg{description} given
        in the \FM{defprimconcept} or \FM{defprimconcept-f}
        definition, possibly extended by absorption of \FM{implies}
        axioms.
%
        \item[\TT{CONCEPT}\hfill -~] The \Arg{description} given in
        the \FM{defconcept} or \FM{defconcept-f} definition.
%
        \item[\TT{ROLE}\hfill -~] A list consisting of the role's
        name, the keyword \TT{:supers} followed by a list of the
        role's \Arg{supers} given in the \FM{defprimrole} or
        \FM{defprimrole-f} definition and the keyword \TT{:transitive}
        followed by \TT{T} if the role is transitive (the last 2 items
        will be omitted if their values are \Nil).
%
        \item[\TT{FEATURE}\hfill -~] A list consisting of the
        attribute's name and the keyword \TT{:supers} followed by a
        list of the attribute's \Arg{supers} given in the
        \FM{defprimattribute} or \FM{defprimattribute-f} definition
        (the last item will be omitted if its value is \Nil).
%
        \end{Arglist}

\end{Function}



\section{Controlling \FaCT's Behavior}

This section describes functions and macros which allow the
classifiers features to be customised and which control performance
profiling.


\begin{Function}{set-features}

\item[\textbf{Description:}\hfill] Enables classifier features.

\item[\textbf{Syntax:}\hfill] (\FM{set-features} \Key{\&rest} \Arg{features})

\item[\textbf{Arguments:}\hfill]%
%
        \begin{Arglist}

        \item[\Arg{features}\hfill -~] Zero or more keywords. The
        feature associated with each supplied keyword is enabled. The
        available keywords are as follows:
        \begin{flushleft}
        \KI{:transitivity} \KI{:concept-eqn}
        \KI{:subset-s-equivalent} \KI{:backjumping} \KI{:obvious-subs}
        \KI{:top-level-caching} \KI{:full-caching} \KI{:blocking}
        \KI{:taxonomic-encoding} \KI{:gci-absorption}
        \KI{:cyclical-definitions} \KI{:auto-configure}
        \KI{:moms-heuristic} \KI{:prefer-pos-lits}
        \KI{:minimise-clashes} \KI{:auto-install-primitives}
        \KI{:auto-install-transitive}
        \end{flushleft}
        By default all features are enabled except:
        \begin{flushleft}
        \KI{:moms-heuristic},
        \KI{:auto-install-primitives} \KI{:auto-install-transitive}
        \end{flushleft}
        \end{Arglist}

\item[\textbf{Return:}\hfill] A list of all enabled features.

\item[\textbf{Remarks:}\hfill] See source code for details of the
effect of each feature.

\item[\textbf{Examples:}\hfill] \TT{(set-features)} \\
        \TT{(set-features :transitivity :backjumping)}

\end{Function}


\begin{Function}{reset-features}

\item[\textbf{Description:}\hfill] Disables classifier features.

\item[\textbf{Syntax:}\hfill] (\FM{reset-features} \Key{\&rest} \Arg{features})

\item[\textbf{Arguments:}\hfill]%
%
        \begin{Arglist}

        \item[\Arg{features}\hfill -~] Zero or more keywords. The
        feature associated with each supplied keyword is disabled.

        \end{Arglist}

\item[\textbf{Return:}\hfill] A list of all enabled features.

\item[\textbf{Remarks:}\hfill] See \FM{set-features} for a list of
features. See source code for details of the effect of each feature.

\item[\textbf{Examples:}\hfill] \TT{(reset-features :transitivity :backjumping)}

\end{Function}


\begin{Function}{features}

\item[\textbf{Description:}\hfill] Prints information about feature settings.

\item[\textbf{Syntax:}\hfill] (\FM{features} \Key{\&optional}
(\Arg{stream} \TT{T}))

\item[\textbf{Arguments:}\hfill]%
%
        \begin{Arglist}

        \item[\Arg{stream}\hfill -~] Optional output stream; default
        is \TT{T} (\Key{*terminal-io*}).

        \end{Arglist}

\item[\textbf{Examples:}\hfill] \TT{(features)}

\end{Function}


\begin{Function}{set-profiling}

\item[\textbf{Description:}\hfill] Controls performance profiling.

\item[\textbf{Syntax:}\hfill] (\FM{set-profiling} \Key{\&key}
(\Arg{level} 1) (\Arg{file} ``profile.out''))

\item[\textbf{Arguments:}\hfill]%
%
        \begin{Arglist}

        \item[\Arg{level}\hfill -~] Keyword which controls the amount
of profiling data which is output:

        \begin{Arglist}

                \item[0\hfill -~] Disables profiling.

                \item[1\hfill -~] Outputs profiling data for
                each concept classification.

                \item[2\hfill -~] Outputs profiling data for
                each subsumption test.

                \item[3\hfill -~] Outputs profiling data for
                each satisfiability test.

        \end{Arglist}

        If omitted, \Arg{level} defaults to 1.


        \item[\Arg{file}\hfill -~] Keyword specifying the name of a
        file to which the profiling data is to be written. If
        explicitly \Nil\ it is written to the \Key{*terminal-io*}
        stream. If omitted, \Arg{file} defaults to
        ``profile.out''.

        \end{Arglist}

\item[\textbf{Remarks:}\hfill] Note that profiling can generate a
        large amount of data, particularly if \Key{level} is $>$1. For
        each satisfiability test the profiler outputs the number of
        backtracks, the maximum model size, the maximum model depth,
        the number of cache accesses, the number of cache hits, the
        CPU time used, the result (\TT{T} or \Nil) and whether
        blocking was triggered (\TT{T} or \Nil). See source code for
        more details.

\item[\textbf{Examples:}\hfill] \TT{(set-profiling)} \\
        \TT{(set-profiling :level 2 :file nil)}

\end{Function}


\begin{Function}{reset-profiling}

\item[\textbf{Description:}\hfill] Disables performance profiling.

\item[\textbf{Syntax:}\hfill] (\FM{reset-profiling})

\item[\textbf{Remarks:}\hfill] Equivalent to \TT{(set-profiling :level 0)}.

\item[\textbf{Examples:}\hfill] \TT{(reset-profiling)}

\end{Function}


\begin{Function}{set-verbosity}

\item[\textbf{Description:}\hfill] Controls the verbosity of the classifier.

\item[\textbf{Syntax:}\hfill] (\FM{set-verbosity} \Key{\&rest} \Arg{features})

\item[\textbf{Arguments:}\hfill]%
%
        \begin{Arglist}

        \item[\Arg{features}\hfill -~] Zero or more keywords. The
        verbosity feature associated with each supplied keyword is enabled. The
        available keywords are as follows:
        \begin{flushleft}
        \KI{:warnings} \KI{:notes} \KI{:synonyms} \KI{:reclassifying}
        \KI{:features} \KI{:rc-counts} \KI{:test-counts}
        \KI{:cache-counts} \KI{:classify-1} \KI{:classify-2}
        \end{flushleft}
        By default, the of enabled verbosity features are:
        \begin{flushleft}
        \KI{:warnings} \KI{:synonyms} \KI{:reclassifying}
        \KI{:rc-counts} \KI{:test-counts}
        \KI{:cache-counts}
        \end{flushleft}

        \end{Arglist}

\item[\textbf{Return:}\hfill] A list of all enabled verbosity features.

\item[\textbf{Remarks:}\hfill] The verbosity setting is temporarily
overridden by the \Arg{verbose} argument to \FM{load-tkb} and the
\Arg{mode} argument to \FM{classify-tkb}. See source code for details
of the effect of each verbosity feature.

\item[\textbf{Examples:}\hfill] \TT{(set-verbosity)} \\
        \TT{(set-verbosity :notes :classify-2)}

\end{Function}


\begin{Function}{reset-verbosity}

\item[\textbf{Description:}\hfill] Reduces the verbosity of the classifier.

\item[\textbf{Syntax:}\hfill] (\FM{reset-verbosity} \Key{\&rest} \Arg{features})

\item[\textbf{Arguments:}\hfill]%
%
        \begin{Arglist}

        \item[\Arg{features}\hfill -~] Zero or more keywords. The
        feature associated with each supplied keyword is disabled. If
        no keywords are supplied, all verbosity features are disabled.

        \end{Arglist}

\item[\textbf{Return:}\hfill] A list of all enabled verbosity features.

\item[\textbf{Remarks:}\hfill] See \FM{set-features} for a list of
verbosity features. See source code for details of the effect of each
feature.

\item[\textbf{Examples:}\hfill] \TT{(reset-verbosity)} \\
        \TT{(reset-verbosity :notes :classify-2)}

\end{Function}



\section{Modal Logic Theorem Proving}

\FaCT\ can also be used as a decision procedure for the propositional
modal logics \Km\ and \Kfour. Although the functions and macros
described in Section~\ref{sec:interface} could be used for this
purpouse, \FaCT\ provides a more convenient functional interface for
performing single satisfiability tests on \Km\ and \Kfour\ formulae
encoded as an \ALC concept description.


\begin{Function}{alc-concept-coherent}

\item[\textbf{Description:}\hfill] Tests the satisfiability of an
        \ALC concept.

\item[\textbf{Syntax:}\hfill] (\FM{alc-concept-coherent} \Arg{description} \Key{\&key}
        (\Arg{k4} \Nil))

\item[\textbf{Arguments:}\hfill]%
%
        \begin{Arglist}

        \item[\Arg{description}\hfill -~] Concept description.

        \item[\Arg{k4}\hfill -~] Keyword which, if non-\Nil, causes
        \Arg{description} to be treated as a \Kfour\ formula (all
        roles are transitive). If omitted it defaults to \Nil.

        \end{Arglist}

\item[\textbf{Return:}\hfill] Returns four values: \TT{T} if
\Arg{description} is satisfiable, \Nil\ otherwise; run time (seconds)
\emph{excluding} concept encoding; number of backtracks; maximum model
size (nodes).

\item[\textbf{Remarks:}\hfill] All user defined concepts, roles and
implications are deleted from the TBox (using \FM{init-tkb}); some or
all of the concepts and roles occuring in \Arg{description} are then
automatically installed in the TBox. The verbosity setting is
temporarily overridden and set to \Nil. Profiling is temporarily
disabled.

\settowidth{\temp}{\TT{(alc-concept-coherent '}}
\item[\textbf{Examples:}\hfill]
\TT{(alc-concept-coherent '(and (some R (some R C1))}\\
\hspace*{\temp}\TT{(all R (not C1))))} \\[0.2 \baselineskip]
\TT{(alc-concept-coherent '(and (some R (some R C1))}\\
\hspace*{\temp}\TT{(all R (not C1))) :k4 T)}

%\parbox[t]{0pt}{\vspace{-(\baselineskip + \parskip)}\begin{tabbing}
%\TT{(alc-concept-coherent} \= \TT{'(and (some R (some R C1))}\\
%       \> \TT{(all R (not C1))))} \\
%\TT{(alc-concept-coherent '(and (some R (some R C1))}\\
%       \> \TT{(all R (not C1))) :k4 T)}
%         \end{tabbing}}

\end{Function}




\newpage
\addcontentsline{toc}{section}{References}
\begin{thebibliography}{BH91}

\bibitem[BH91]{Baader91a}
F.~Baader and B.~Hollunder.
\newblock {KRIS}: Knowledge representation and inference system.
\newblock {\em SIGART Bulletin}, 2(3):8--14, 1991.

\bibitem[GS96]{Giunchiglia96b}
F.~Giunchiglia and R.~Sebastiani.
\newblock A {SAT}-based decision procedure for $\mathcal{ALC}$.
\newblock In L.~C. Aiello, J.~Doyle, and S.~Shapiro, editors, {\em Principals
  of Knowledge Representation and Reasoning: Proceedings of the Fifth
  International Conference (KR'96)}, pages 304--314. Morgan Kaufmann, November
  1996.

\bibitem[HS97]{Hustadt97a}
U.~Hustadt and R.~A. Schmidt.
\newblock On evaluating decision procedures for modal logic.
\newblock Technical Report MPI-I-97-2-003, Max-Planck-Institut F\"ur
  Informatik, February 1997.

\end{thebibliography}

%\bibliographystyle{alpha}
%\bibliography{../../bibtex/all}

\printindex

\end{document}
