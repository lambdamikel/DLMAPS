\documentclass[12pt]{article}
\usepackage{epsfig}

\title{Experimental Results}
\author{}
\date{}

\setlength{\parindent}{0pt}
\setlength{\parskip}{1ex plus 0.5ex minus 0.2ex}
%\setlength{\topmargin}{0pt}
\setlength{\headheight}{0pt}
\setlength{\headsep}{0pt}
\setlength{\oddsidemargin}{0pt}
\setlength{\evensidemargin}{0pt}
\setlength{\textwidth}{6.26in}
\setlength{\textheight}{9.693in}

\newcommand{\FaCT}{\textrm{FaCT}}

\hyphenation{data-base tab-leaux}
\sloppy
%
%********************************************************************
%
\begin{document}

\maketitle

\section{Giunchiglia \& Sebastiani Results}

The graphs in Figures~\ref{table:ps0}, \ref{table:ps2},
\ref{table:ps4}, \ref{table:ps6}, \ref{table:ps8} and \ref{table:ps10}
plot the 50th, 60th, 70th, 80th, 90th and 100th percentile
satisfiability times for \FaCT\ using random data generated by the
Giunchiglia \& Sebastiani algorithm~\cite{Giunchiglia96b} with the
following parameters:

\begin{center}\begin{tabular}{|llllll|llllll|}
\hline
 & \multicolumn{1}{c}{$N$} & 
\multicolumn{1}{c}{$M$} & 
\multicolumn{1}{c}{$K$} & 
\multicolumn{1}{c}{$D$} & 
\multicolumn{1}{c|}{$P$} & & 
\multicolumn{1}{c}{$N$} & 
\multicolumn{1}{c}{$M$} & 
\multicolumn{1}{c}{$K$} & 
\multicolumn{1}{c}{$D$} & 
\multicolumn{1}{c|}{$P$} \\
\hline
\textbf{PS0} & 5 & 1 & 3 & 2 & 0.5 & \textbf{PS6} & 4 & 4 & 3 & 2 & 0.5 \\
\textbf{PS1} & 3 & 1 & 3 & 5 & 0.5 & \textbf{PS7} & 4 & 5 & 3 & 2 & 0.5 \\
\textbf{PS2} & 3 & 1 & 3 & 4 & 0.5 & \textbf{PS8} & 4 & 10 & 3 & 2 & 0.5 \\
\textbf{PS3} & 3 & 1 & 3 & 3 & 0.5 & \textbf{PS9} & 4 & 20 & 3 & 2 & 0.5 \\
\textbf{PS4} & 3 & 1 & 3 & 2 & 0.5 & \textbf{PS10} & 8 & 1 & 3 & 2 & 0.5 \\
\textbf{PS5} & 4 & 1 & 3 & 2 & 0.5 & \textbf{PS11} & 10 & 1 & 3 & 2 & 0.5 \\
\hline
\end{tabular}\end{center}


\begin{figure}[p]\begin{center}\begin{tabular}{cc}
\textbf{PS0} & \textbf{PS1} \\
\epsfig{file=../ps0-2d.eps,width=0.45\linewidth} &
\epsfig{file=../ps1-2d.eps,width=0.45\linewidth}
\end{tabular}
\caption{Parameter settings PS0 and PS1}\label{table:ps0}
\end{center}\end{figure}

\begin{figure}[p]\begin{center}\begin{tabular}{cc}
\textbf{PS2} & \textbf{PS3} \\
\epsfig{file=../ps2-2d.eps,width=0.45\linewidth} &
\epsfig{file=../ps3-2d.eps,width=0.45\linewidth}
\end{tabular}
\caption{Parameter settings PS2 and PS3}\label{table:ps2}
\end{center}\end{figure}

\begin{figure}[p]\begin{center}\begin{tabular}{cc}
\textbf{PS4} & \textbf{PS5} \\
\epsfig{file=../ps4-2d.eps,width=0.45\linewidth} &
\epsfig{file=../ps5-2d.eps,width=0.45\linewidth}
\end{tabular}
\caption{Parameter settings PS4 and PS5}\label{table:ps4}
\end{center}\end{figure}

\begin{figure}[p]\begin{center}\begin{tabular}{cc}
\textbf{PS6} & \textbf{PS7} \\
\epsfig{file=../ps6-2d.eps,width=0.45\linewidth} &
\epsfig{file=../ps7-2d.eps,width=0.45\linewidth}
\end{tabular}
\caption{Parameter settings PS6 and PS7}\label{table:ps6}
\end{center}\end{figure}

\begin{figure}[p]\begin{center}\begin{tabular}{cc}
\textbf{PS8} & \textbf{PS9} \\
\epsfig{file=../ps8-2d.eps,width=0.45\linewidth} &
\epsfig{file=../ps9-2d.eps,width=0.45\linewidth}
\end{tabular}
\caption{Parameter settings PS8 and PS9}\label{table:ps8}
\end{center}\end{figure}

\begin{figure}[p]\begin{center}\begin{tabular}{cc}
\textbf{PS10} & \textbf{PS11} \\
\epsfig{file=../ps10-2d.eps,width=0.45\linewidth} &
\epsfig{file=../ps11-2d.eps,width=0.45\linewidth}
\end{tabular}
\caption{Parameter settings PS10 and PS11}\label{table:ps10}
\end{center}\end{figure}


\subsection{Varying Individual Parameters}

The graphs in Figure~\ref{table:varying} show the effect of varying one
of the random generation parameters while the other parameters remain
fixed.

\begin{figure}[p]\begin{center}{\setlength{\tabcolsep}{0pt}\begin{tabular}{lll}
%
1: Varying N:		& 2: Varying M:		& 3: Varying D: \\
M=1, D=2, P=0.5,	& N=4, D=2, P=0.5,	& N=3, M=1, P=0.5, \\
N=3, 4, 5, 8, 10	& M=1, 2, 5, 10, 20	& D=2, 3, 4, 5 \\[2pt]
\multicolumn{3}{c}{\textbf{Median Satisfiability Times (s) -v- L/N}} \\
\multicolumn{1}{c}{\epsfig{file=../varyN-2d.eps,width=0.33\linewidth}} &
\multicolumn{1}{c}{\epsfig{file=../varyM-2d.eps,width=0.33\linewidth}} &
\multicolumn{1}{c}{\epsfig{file=../varyD-2d.eps,width=0.33\linewidth}}\\[2pt]
\multicolumn{3}{c}{\textbf{Median Backtracking Search Space -v- L/N}} \\
\multicolumn{1}{c}{\epsfig{file=../varyN-bt.eps,width=0.33\linewidth}} &
\multicolumn{1}{c}{\epsfig{file=../varyM-bt.eps,width=0.33\linewidth}} &
\multicolumn{1}{c}{\epsfig{file=../varyD-bt.eps,width=0.33\linewidth}}\\[2pt]
\multicolumn{3}{c}{\textbf{Probability of Satisfiability -v- L/N}} \\
\multicolumn{1}{c}{\epsfig{file=../varyN-ps.eps,width=0.33\linewidth}} &
\multicolumn{1}{c}{\epsfig{file=../varyM-ps.eps,width=0.33\linewidth}} &
\multicolumn{1}{c}{\epsfig{file=../varyD-ps.eps,width=0.33\linewidth}}
\end{tabular}}
%
\caption{Varying parameters N, M and D}\label{table:varying}
\end{center}\end{figure}



\section{Hustadt \& Schmidt Results}

The graphs in Figure~\ref{table:ps12} plot the 50th, 60th, 70th, 80th,
90th and 100th percentile satisfiability times for \FaCT\ using
random data generated by the Hustadt \& Schmidt
algorithm~\cite{Hustadt97a} with the following parameters:

\begin{center}\begin{tabular}{|llllll|llllll|}
\hline
 & \multicolumn{1}{c}{$N$} & 
\multicolumn{1}{c}{$M$} & 
\multicolumn{1}{c}{$K$} & 
\multicolumn{1}{c}{$D$} & 
\multicolumn{1}{c|}{$P$} & & 
\multicolumn{1}{c}{$N$} & 
\multicolumn{1}{c}{$M$} & 
\multicolumn{1}{c}{$K$} & 
\multicolumn{1}{c}{$D$} & 
\multicolumn{1}{c|}{$P$} \\
\hline
\textbf{PS12} & 4 & 1 & 3 & 1 & 0 & \textbf{PS13} & 6 & 1 & 3 & 1 & 0 \\
\hline
\end{tabular}\end{center}


\begin{figure}[p]\begin{center}\begin{tabular}{cc}
\textbf{PS12} & \textbf{PS13} \\
\epsfig{file=../ps12-2d.eps,width=0.45\linewidth} &
\epsfig{file=../ps13-2d.eps,width=0.45\linewidth}
\end{tabular}
\caption{Parameter settings PS12 and PS13}\label{table:ps12}
\end{center}\end{figure}

\newpage

\begin{thebibliography}{GS96}

\bibitem[GS96]{Giunchiglia96b}
F.~Giunchiglia and R.~Sebastiani.
\newblock A {SAT}-based decision procedure for $\mathcal{ALC}$.
\newblock In L.~C. Aiello, J.~Doyle, and S.~Shapiro, editors, {\em Principals
  of Knowledge Representation and Reasoning: Proceedings of the Fifth
  International Conference (KR'96)}, pages 304--314. Morgan Kaufmann, November
  1996.

\bibitem[HS97]{Hustadt97a}
U.~Hustadt and R.~A. Schmidt.
\newblock On evaluating decision procedures for modal logic.
\newblock Technical Report MPI-I-97-2-003, Max-Planck-Institut F\"ur
  Informatik, February 1997.

\end{thebibliography}
%\bibliographystyle{alpha}
%\bibliography{../../bibtex/all}

\end{document}
